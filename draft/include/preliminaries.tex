\section{Preliminaries and Definitions}
{\color{red} Talk about KBs. Describe what entities and literals are}

\subsection{Language} \label{sec:language}
{\color{red} Horn Rule. Given the definition of a \emph{valid} rule (each variables appearing twice and each variable connected transitively to every other variable) and also give the definition of a valid body (a and b at least once, other variables twice).
Extension of predicates with inequalities.
Define instantiation of an atom.}


A Horn Rule $r$ has the form $A_1 \wedge A_2 \wedge \cdots \wedge A_n \Rightarrow \atom{r}{a}{b}$, where  $A_1 \wedge A_2 \wedge \cdots \wedge A_n = r_{body}$ is the \emph{body} of the rule.

\subsection{Coverage}
Given a pair of entities $(x,y)$ from the KB and a Horn Rule $r$, we say that $r_{body}$ \emph{covers} $(x,y)$ if
$(x,y) \models r_{body}$. In other words, given a Horn Rule $r = r_{body} \Rightarrow \atom{r}{a}{b}$, $r_{body}$ covers a pair of entities $(x,y)$ iff $r_{body}$ can be instantiated over the KB by substituting $a$ with $x$ and $b$ with $y$. Given a set of pair of entities $E = \{(x_1,y_1),(x_2,y_2),\cdots,(x_n,y_n)\}$ and a rule $r$, we denote by $C_r(E)$ the \emph{coverage} of $r_{body}$ over $E$ as the set of elements in $E$ covered by $r$, $C_r(E)=\{(x,y) \in E | (x,y) \models r_{body}\}$.

Given the body $r_{body}$ of a Horn Rule $r$, we denote by $r^{*}_{body}$ the \emph{unbounded body} of $r$. The unbounded body of a rule is obtained by substituting each atom in $r$ that contains either variable $a$ or $b$ with a new atom where the other variable that is not $a$ or $b$ 
is substituted with another unique variable. As an example, given $r_{body} = \atom{rel}{a}{b}$, $r^{*}_{body} = \atom{rel}{a}{v_1} \wedge \atom{rel}{v_2}{b}$.
\paolo{I suggest to have $rel_3(a,b)$ to avoid the confusion raised by cartesian product} 
\stefano{Better now?}\paolo{to me this is still unclear, the way that is written it applies for all combinations of a and b universal variables, no matter if they are related or not. Therefore it seems the cartesian product. The same seems to be stated in the following example. Adding a small instance in the intro will probably clarify things}
Given a set of pair of entities $E = \{(x_1,y_1),(x_2,y_2),\cdots,(x_n,y_n)\}$ and a rule $r$, we denote by $U_r(E)$ the \emph{unbounded coverage} of $r^{*}_{body}$ over $E$ as the set of elements in $E$ covered by $r^{*}_{body}$, $U_r(E)=\{(x,y) \in E | (x,y) \models r^{*}_{body}\}$.

\begin{myExample}
	Given the negative rule $r$ of Example~\ref{ex:intro} and a KB $K$, we denote by $E$ the set of all possible pairs of entities in $K$. The coverage of $r$ over $E$ $(C_r(E))$ is the set of all pairs of entities $(x,y)$ where both $x$ and $y$ have the \emph{\texttt{birthDate}} information and $x$ is born after $y$, while the unbounded coverage of $r$ over $E$ $(U_r(E))$ is the set of all pairs of entities $(x,y)$ where both $x$ and $y$ have the \emph{\texttt{birthDate}} information, no matter what the relation is between the two birth dates. 
\end{myExample}

The unbounded coverage is essential to distinguish between missing and inconsistent information: if for a pair of entities $(x,y)$ the \texttt{birthDate} information is missing from the KB for either $x$ or $y$, we cannot say whether $x$ was born before or after $y$, therefore we cannot be sure that the negative rule of Example~\ref{ex:intro} does not cover $(x,y)$. Instead if both $x$ and $y$ have the \texttt{birthDate} information and $x$ was born before $y$, we can affirm that the negative rule of Example~\ref{ex:intro} does not cover $(x,y)$. Given that modern KBs are largely incomplete (REFERENCE), discriminating between missing and conflicting information becomes of paramount importance.

Similarly, the coverage and the unbounded coverage for a set of rules $R=\{r_1,r_2,\cdots,r_n\}$ is the union of individual coverages:

$$C_R(E) = \bigcup \limits_{r \in R} C_r(E) \qquad U_R(E) = \bigcup \limits_{r \in R} U_r(E) $$

Our problem is the discovery of positive (negative) rules for an input given relation. We uniquely identify a relation with two different sets of pair of entities.
	$G$ -- \emph{generation set}. $G$ contains good examples for the relation that we are trying to discover ($G$ contains examples of parents and children if we are discovering positive rules for a child relation).
	$V$ -- the validation set. $V$ contains counter examples for the target relation (pairs of people that are not in a child relation).
We will explain in Section~\ref{sec:examples_gen} how to generate these two sets for a given relation.
Note that our approach is not less generic than those for mining rules for the entire KB (e.g.,~\cite{abedjan2014amending,galarraga2015fast}): it is true that we require a target relation as input, however we can generically apply such setting for every relation in the KB and compute rules for each of them.


We can now formalize the \emph{exact discovery problem}.
Given a KB $K$, a set of pair of entities $G$, a set of pair of entities $V$, and a universe of rules $R$, 
a solution for the \emph{exact discovery problem} is a subset $R'$ of $R$  such that:
$$R_{opt}=\underset{|R'|}{\operatorname{argmin}}(R'|(C_{R'}(G) = G) \wedge (C_{R'}(V) \cap V = \emptyset) )$$
The ideal solution is a set of rules that covers all examples in $G$, and none of the examples in $V$. Note that given a pair of entities $(x,y)$, we can always generate a Horn Rule whose body covers only $(x,y)$ by assigning variable $a$ to $x$ and variable $b$ to $y$.

Unfortunately, since the solution is not allowed to cover any element in $V$, in the worst case the exact solution may be a set of rules s.t. each rule covers only one example in $G$, making such set of rules difficult to use.

\subsection{Weight Function}
In order to allow flexibility and errors in both $G$ and $V$, we drop the strict requirement of not covering any element of $V$. However, since covering elements in $V$ is an indication of potential errors, we want to limit the coverage over $V$ to the minimum possible. We therefore define a $weight$ to be associated with a rule.


Given a KB $K$, two sets of pair of entities $G$ and $V$ from $K$ where $G \cap V = \emptyset$, and a Horn Rule $r$, the weight of $r$ is defined as follow:
\begin{equation} \label{eq:weight_fun}
	w(r) = \alpha \cdot (1-\frac{\mid C_{r}(G)\mid}{\mid G \mid}) +\beta \cdot (\frac{\mid C_{r}(V) \mid}{\mid U_{r}(V)\mid})  +\gamma \cdot (1-\frac{\mid U_{r}(V)\mid}{\mid V \mid})
\end{equation}
with $\alpha,\beta,\gamma \in [0,1]$ and $\alpha + \beta + \gamma = 1$. The weight is a value between $0$ and $1$ that captures the \emph{goodness} of a rule w.r.t. $G$ and $V$: the better the rule, the lower the weight -- perfect rule would have a weight of $0$. The weight is made of three components normalized by the three parameters $\alpha,\beta,\gamma$.
\begin{inparaenum}[\itshape(i)]
	\item The first component captures the coverage over the generation set $G$ -- the ratio between the coverage of $r$ over $G$ and $G$ itself. Note that if $r$ covers all elements in $G$, then this component is $0$ because of the subtraction from $1$.
	\item The second component aims to quantify potential errors of $r$, or rather the coverage over $V$. The coverage over $V$ is not divided by total elements in $V$, because 
	for those elements in $V$ that do not have relations stated in $r$ we cannot be sure that such elements are not covered by $r$. Thus we divide the coverage over $V$ by the unbounded coverage of $r$ over $V$. Ideally this number is close to $0$.
	\item The last element of the weight captures how many elements of $V$ have the information stated by relations in $r$. The more elements in $V$ are unbounded covered by $r$, the better we can judge the rule w.r.t. $V$. This element is close to $0$ when $r$ unbounded covers many elements of $V$. 
\end{inparaenum}
The parameters $\alpha,\beta,\gamma$ are used to give more relevance to some components. We would set a high $\beta$ if we want to discover rules with high precision that identifies few mistakes, or we would set a high $\alpha$ if we are more interested in recall and the discovered rules should identify as many examples as possible.

\begin{myExample}
	W.r.t. the negative rule $r$ of Example~\ref{ex:intro}, given two sets of pair of entities $G$ and $V$ the three components of $w_r$ are computed as follow:
	\begin{inparaenum}[\itshape(i)]
		\item the first component is computed as 1 minus number of pairs $(x,y)$ in $G$ where
		$x$ is born after $y$ divided by the number of elements in $G$;
		\item the second component is the ratio between number of pairs $(x,y)$ in $V$ where $x$ is born after $y$ and number of pairs $(x,y)$ in $V$ where the date of birth (for both $x$ and $y$) is available in the KB;
		\item the last component is computed as 1 minus number of pairs $(x,y)$ in $V$ where the date of birth (for both $x$ and $y$) is available in the KB divided by the total number of elements in $V$.
	\end{inparaenum}
\paolo{I think this would be more useful by referring to data in an example with micro KB}
	\end{myExample}


Similarly, the weight for a set of rules $R$ is defined as:
\begin{equation*}
w(R) = \alpha \cdot (1-\frac{\mid C_{R}(G)\mid}{\mid G \mid}) +\beta \cdot (\frac{\mid C_{R}(V) \mid}{\mid U_{R}(V)\mid})  +\gamma \cdot (1-\frac{\mid U_{R}(V)\mid}{\mid V \mid})
\end{equation*}

Assigning a weight to one or multiple rules allows us to take into consideration an important aspect of modern KBs: the presence of errors. We will show in the experimental evaluation that very rarely rules have a $0$ coverage over the validation set, and very often good rules have a significant coverage over $V$. The exact discovery problem implies the absence of errors in the input KB, unfortunately such assumption is too strong for modern KBs that are automatically built(REFERENCE).

\subsection{Problem Definition}
We can now state the approximate version of the problem.


Given a KB $K$, two sets of pair of entities $G$ and $V$ from $K$ where $G \cap V = \emptyset$, a universe of rules $R$, and a $w$ weight function for $R$,
a solution for the \emph{approximate discovery problem} is a subset $R'$ of $R$  such that:

$$R_{opt}=\underset{w(R')}{\operatorname{argmin}}(R'|R'(G) = G)$$


We can map this problem to the well-known weighted set cover problem, which is proven to be a NP-Complete problem~\cite{chvatal1979greedy}, where the universe is $G$ and the sets are all the possible rules defined in $R$.

Since we want to minimize the total weight of the output rules, the approximate version of the discovery problem aims to cover all elements in $G$, and as few as possible elements in $V$. Since for each element $(x,y)$ in $G$ there always exists a rule that covers only $(x,y)$ (single-instance rule), an optimal output is always guaranteed to exist. We expect such output to be made of some rules that covers more than one example in $G$, and the remaining examples in $G$ to be covered by single-instance rules. In the best-case scenario a single rule covers all elements in $G$ and none of the elements in $V$.

Section~\ref{sec:greedy_alg} will describe a greedy polynomial algorithm to find a good solution for our problem. \paolo{Perhaps we should clarify the role of the queries in the bodies vs the role of the sets, and the fact that we do not have all the queries materialized because of their large number }



