\section{Preliminaries and Definitions}
{\color{red} Talk about KBs.}

\subsection{Language}
{\color{red} Horn Rule, with the restriction of having each variables appearing twice.
Extension of predicates with inequalities.}

A Horn Rule $r$ has the form $A_1 \wedge A_2 \wedge \cdots A_n \Rightarrow r(a,b)$, where  $A_1 \wedge A_2 \wedge \cdots \wedge A_n = r_{body}$ is the \emph{body} of the rule.

\subsection{Coverage}
Given a pair of entities $(x,y)$ from the KB and a Horn Rule $r$, we say that $r_{body}$ \emph{covers} $(x,y)$ if
$(x,y) \models r_{body}$. In other words, given a Horn Rule $r = r_{body} \Rightarrow r(a,b)$, $r_{body}$ covers a pair of entities $(x,y)$ iff $r_{body}$ can be instantiated over the KB by substituting $a$ with $x$ and $b$ with $y$. Given a set of pair of entities $E = \{(x_1,y_1),(x_2,y_2),\cdots,(x_n,y_n)\}$ and a rule $r$, we denote by $C_r(E)$ the \emph{coverage} of $r_{body}$ over $E$ as the set of elements in $E$ covered by $r$, $C_r(E)=\{(x,y) \in E | (x,y) \models r_{body}\}$.

Given the body $r_{body}$ of a Horn Rule $r$, we denote by $r^{*}_{body}$ the \emph{unbounded body} of $r$. The unbounded body of a rule is obtained by substituting each atom in $r$ that contains either variable $a$ or $b$ with a new atom where the other variable that is not $a$ or $b$ 
is substituted with another unique variable. As an example, given $r_{body} = rel_1(a,b)$, $r^{*}_{body} = rel_1(a,v_1) \wedge rel_1(v_2,b)$. 
\paolo{I suggest to have $rel_3(a,b)$ to avoid the confusion raised by cartesian product} 
\stefano{Better now?}
Given a set of pair of entities $E = \{(x_1,y_1),(x_2,y_2),\cdots,(x_n,y_n)\}$ and a rule $r$, we denote by $U_r(E)$ the \emph{unbounded coverage} of $r^{*}_{body}$ over $E$ as the set of elements in $E$ covered by $r^{*}_{body}$, $U_r(E)=\{(x,y) \in E | (x,y) \models r^{*}_{body}\}$.

\begin{myExample}
	Given the rule \emph{$r=\texttt{hasChild(}a,v_0\texttt{)} $ $\wedge \texttt{hasChild(}b,v_0\texttt{)}$} and a KB $K$, we denote by $E$ the set of all possible pairs of entities in $K$.
	 The coverage of $r$ over $E$ $(C_r(E))$ is the set of all pairs of entities $(x,y)$ where both $x$ and $y$ are in relation \emph{\texttt{hasChild}} with the same entity $v_0$, while the unbounded coverage of $r$ over $E$ $(U_r(E))$ is the set of all pairs of entities $(x,y)$ where $x$ is in relation \emph{\texttt{hasChild}} with an entity $v_1$ and $y$ is in relation \emph{\texttt{hasChild}} with an entity $v_2$, and not necessarily $v_1 = v_2$. 
\end{myExample}

\stefano{Explain why unbounded coverage is important}

Similarly, the coverage and the unbounded coverage for a set of rules $R=\{r_1,r_2,\cdots,r_n\}$ is the union of individual coverages:

$$C_R(E) = \bigcup \limits_{r \in R} C_r(E) \qquad U_R(E) = \bigcup \limits_{r \in R} U_r(E) $$

We can now formalize the \emph{exact discovery problem}.
Given a KB $K$, a set of pair of entities $G$, a set of pair of entities $V$, and a universe of rules $R$, 
a solution for the \emph{exact discovery problem} is a subset $R'$ of $R$  such that:
$$R_{opt}=\underset{|R'|}{\operatorname{argmin}}(R'|(C_{R'}(G) = G) \wedge (C_{R'}(V) \cap V = \emptyset) )$$
$G$ is the \emph{generation set}, which contains good examples for the rule that we are trying to discover ($G$ contains examples of married couples if we are discovering rules for a spouse relation). $V$ is the validation set, which contains counter examples for the target rule (pairs of people that are not in a married relation). The ideal solution is a set of rules that covers all examples in $G$, and none of the examples in $V$. Note that given a pair of entities $(x,y)$, we can always generate a Horn Rule whose body covers only $(x,y)$ by assigning variable $a$ to $x$ and variable $b$ to $y$.

Unfortunately, since the solution is not allowed to cover any element in $V$, in the worst case the exact solution may be a set of rules s.t. each rule covers only one example in $G$, making such set of rules difficult to use.

\subsection{Scoring Function}
In order to allow flexibility and errors in both $G$ and $V$, we drop the strict requirement of not covering any element of $V$. However, since covering elements in $V$ is an indication of potential errors, we want to limit the coverage over $V$ to the minimum possible. We therefore define a $weight$ to be associated with a rule.


Given a KB $K$, two sets of pair of entities $G$ and $V$ from $K$ where $G \cap V = \emptyset$, and a Horn Rule $r$, the weight of $r$ is defined as follow:
\begin{equation*}
	w(r) = \alpha \cdot (1-\frac{\mid C_{r}(G)\mid}{\mid G \mid}) +\beta \cdot (\frac{\mid C_{r}(V) \mid}{\mid U_{r}(V)\mid})  +\gamma \cdot (1-\frac{\mid U_{r}(V)\mid}{\mid V \mid})
\end{equation*}

\paolo{if we want to minimize it, it should be defined as a cost function}
\stefano{weight ok?}
\paolo{here goes the description of the three intuition behind the three components, why they are needed, etc. Show with example introduce in intro}

Similarly, the weight for a set of rules $R$ is defined as:
\begin{equation*}
w(R) = \alpha \cdot (1-\frac{\mid C_{R}(G)\mid}{\mid G \mid}) +\beta \cdot (\frac{\mid C_{R}(V) \mid}{\mid U_{R}(V)\mid})  +\gamma \cdot (1-\frac{\mid U_{R}(V)\mid}{\mid V \mid})
\end{equation*}

\subsection{Problem Definition}
We can now state the approximate version of the problem.


Given a KB $K$, two sets of pair of entities $G$ and $V$ from $K$ where $G \cap V = \emptyset$, a universe of rules $R$, and a $w$ weight function for $R$,
a solution for the \emph{approximate discovery problem} is a subset $R'$ of $R$  such that:

$$R_{opt}=\underset{w(R')}{\operatorname{argmin}}(R'|R'(G) = G)$$


We can map this problem to the well-known weighted set cover problem, which is proven to be an NP-Complete problem~\cite{chvatal1979greedy}, where the universe is $G$ and the sets are all the possible rules defined in $R$.

\stefano{Discuss what the approximate version of the problem is trying to accomplish: cover all elements in G, and as few as possible in V. Discuss that in the worst case there will not exists good rules that cover more than one element in V and few elements in G. In those cases for those elements in V that cannot be covered, there will be a single rule for each element.}

Section~\ref{sec:greedy_alg} will describe a greedy polynomial algorithm to find a good solution for our problem.



