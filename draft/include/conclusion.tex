\section{Conclusion}

%We have designed and implemented 
In this paper we presented \krd, a disk-based rule discovery system that mines both positive and negative declarative rules on RDF KBs. Positive rules identify new valid facts for the KB, while negative rules are useful to identify errors. 
We have experimentally demonstrated that our approach generates concise sets of meaningful rules with high precision,
is scalable, and can work with KBs of large size. 
Also, we shown that negative rules not only identify potential errors in the KBs, but also discover new false facts that can serve as representative training data to ML algorithms.

Interesting open questions are related to the support of interactive discovery of the rules. It is not clear if and how is possible to drastically reduce the runtime of the discovery with sampling of the input training instances while not compromising on the quality of the mined rules. 
%We also plan to discern between a positive and a negative rule predicting the same set of facts from the KB as this implies that only one of the rules holds. 
%This can also be an indirect approach to handle the incompleteness of the KBs that remains a huge challenge.
%In the future, we want to extend \krd to mine rules from free web text which is not as structured as KBs. 
%We would like 
Another interesting future direction is to discover more expressive rules that can exploit temporal information through smarter analysis of literals~\cite{abedjan2015temporal}. For instance, ``if two people are not born in the same century, then they cannot be married" is an example rule that requires non-trivial analysis of temporal information. 
%Extend rules with temporal and local information.

%Combine positive and negative rules: if a positive and negative rule identifies same triples, then one of them must be wrong.

%Introduce functions and smarter literals comparison: if two people are not born in the same century, then they cannot be married.