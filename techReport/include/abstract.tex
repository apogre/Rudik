\begin{abstract}
	We present \krd, a system for the discovery of declarative rules over knowledge-bases (KBs).
	\krd output is not limited to rules that rely on ``positive'' relationships between entities, such as ``if two persons have the same parent, they are siblings'', as in traditional  constraint mining for KBs. On the contrary, it discovers also negative rules, i.e., patterns that lead to contradictions in the data, such as ``if two persons are married, one cannot be the child of the other". While the former class is fundamental to infer new relationships in the KB, the latter class is crucial for other tasks, such as error detection in data cleaning, or the creation of negative examples to bootstrap learning algorithms.
	%
	The algorithm to discover positive and negative rules is designed with three main requirements: 
	\begin{inparaenum}[\itshape(i)]
		\item enlarge the {\em expressive power} of the rule language to 
		%comparisons among constants, including disequalities, and therefore 
		obtain complex rules and wide coverage of the facts in the KB, 
		\item allow the discovery of {\em approximate} rules to be robust to errors and incompleteness in the KB, 
		\item use disk-based algorithms, effectively 
		%dropping the assumption that the KB has to fit in memory and
		%have acceptable performance and 
		enabling the mining of large KBs in commodity machines.
	\end{inparaenum}
	%, therefore . 
	%To guarantee that the entire search space is explored, we formalize the mining problem as an incremental graph exploration. 
	%Our novel search strategy is coupled with a number of optimization techniques to further prune and efficiently maintain the search space. 
	%
	%Finally, in contrast with traditional ranking of rules based on a measure of support, we propose a new approach inspired by set cover to identify the subset of useful rules to be exposed to the user. 
	%
	We have conducted extensive experiments using real-world KBs to show that \krd outperforms previous proposals in terms of efficiency and that it discovers more effective rules for the application at hand.
\end{abstract}